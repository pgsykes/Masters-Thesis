\chapter{Results}
\label{chap:results}

\section{Introduction}

In the previous chapter, we described the parameters of a commercial WDM optical fiber communication system with a frequency signal that is located between two data channels, as shown in Fig.~\ref{fig:system}. Given reasonable system parameters, we showed that the dominant optical impairment is XPM, given by Eq.~\ref{eq:phasedistort}, and all other optical impairments can be made negligible.

It follows that the phase distortion depends on the length of fiber, the frequency separation between the frequency signal and the data channels, and the power of the neighboring data channels as they change over the course of their propagation. We will now vary these parameters and investigate their effect on the stability of the frequency signal.

We choose a value for the data channel power that is typical in commercial optical communication systems. These values are chosen to minimize nonlinear distortion in the data signals \cite{Agrawal2013, agrawal2012fiber}. Hence, we neglect the nonlinear distortion of the data signals when calculating $\phi(z,t)$ and focus on the effect of dispersion. The evolution of the data signal is then easily obtained in the Fourier domain, and we find
%
\begin{equation} \label{eq:datasol}
u_d(z,T) = \frac{1}{2\pi} \int_{-\infty}^{\infty} U_d(0, \omega')\exp\left(-\frac{\alpha}{2} z + i\delta\omega' z + \frac{i}{2}\beta_2\omega'^2z-i\omega' t\right) d\omega' ,
\end{equation}
%
where $U_d(0,\omega)$ is the Fourier spectrum of the data signal at $z=0$ defined as
\begin{equation}
U_d(0,\omega) = \int_{-\infty}^{\infty} u_d(0, t')\exp(i\omega t')dt'.
\end{equation}

We begin by investigating the distribution of the power of the data signal, $|u_d|^2$, as a function of length $z$. The on-off-keyed nonreturn-to-zero (OOK-NRZ) symbols of the data signal change over the length of the fiber due to dispersion, self-phase modulation, and attenuation. We first study a system in which attenuation is neglected. We then add the effect of attenuation. Finally, we study the system behavior as the group velocity difference due to the frequency separation between the neighboring data channels and frequency signal increases.

\section{Simulation Parameters}

The simulated data signal is a $2^{10}-1$ pseudorandom binary string (PRBS) \cite{PRBS} that is OOK-NRZ modulated with optical power of $1$ mW with periodic boundary conditions. The PRBS is used because it simplifies our computations and it has properties that resemble a random sequence (its autocorrelation function is approximately a delta function and there are almost equal number of $0$s and $1$s). The length of the string is chosen so that the string doesn't repeat as the data channel travels through a fixed time point in the frequency signal.

The fiber has an attenuation of $\alpha = 0.2$ dB/km, group velocity dispersion $\beta_2 = -22$ ps$^2$/km, and Kerr nonlinearity $\gamma = 1.3$ W$^{-1}$km$^{-1}$. The data signal has a central wavelength of $1530$ nm with inverse group velocity difference $\delta = 1$ ps/km relative to the frequency signal. These parameters are typical for optical fiber communication systems \cite{agrawal2012fiber,Agrawal2013}. We will vary some of these parameters in the following sections as we study the changes in the XPM-induced phase distortion.


\section{Without Attenuation}

We first neglect attenuation in order to provide a baseline against which to determine its effect.

During propagation, the optical power in each bit of the data signal spreads outside of its time slot into the time slots of its neighbors. After some long distance, the expected power in each time slot will become the same. Therefore, the variance of the data signal's optical power is expected to decrease as a function of fiber length, although there will be statistical fluctuations in the pseudo-random signal that we are using. Figure~\ref{fig:NACalcVar} shows the data signal power variance as the distance varies up to $800$ km. Since there is no attenuation, the mean of the data signal power is constant. Though not shown in the figure, longer propagation distances yield a variance on the order of $1.8 \times 10^{-7}$ W$^2$.
%
\begin{figure}[htb]
	\centering
	\includegraphics[scale=0.9]{img/NACalcVar}
	\caption{Data channel optical power variance vs. fiber length} \label{fig:NACalcVar}
\end{figure}
%


The phase shift $\phi$ grows as a function of distance because the frequency signal experiences cross-talk from the data signal, which accumulates over the propagation length. Figures \ref{fig:NAPhiMean}a and \ref{fig:NAPhiVar}b show the mean and variance of the phase shift of the frequency signal due to XPM. The mean of $\phi$ grows linearly with respect to the fiber length because without attenuation the average energy in the data signal is constant. As a consequence, the mean additive phase error can be compensated.
%
\begin{figure}[htb]
	\begin{tabular}{c c}
		\includegraphics[width=0.5\linewidth]{img/NAPhiMean} & \includegraphics[width=0.5\linewidth]{img/NAPhiVar} \\
		(a) & (b)
	\end{tabular}
	\caption{(a)\label{fig:NAPhiMean} Mean of $\phi$ vs. fiber length (b)\label{fig:NAPhiVar} Variance of $\phi$ vs. fiber length}
\end{figure}
%

We will quantify the phase deviation using the measures that we introduced in Chapter \ref{chap:time_stability}. We first consider the first structure equation, $D^{(1)}_\phi = \langle [\phi(t+\tau) - \phi(t)]^2 \rangle$, which represents the mean phase accumulation. As we discussed in Chapter \ref{chap:time_stability}, the structure functions are related to the autocorrelation function. A typical data signal is a collection of apparently random bits that are uncorrelated with each other. As the data signal propagates through the fiber, the optical energy associated with each bit occupies a larger amount of time due to dispersion, so that the amount of time in which a bit is correlated with itself increases. Figure \ref{fig:NAPhaseStability} shows $\left[D^{(1)}_\phi\right]^{1/2}$ at different lengths. The phase deviation becomes constant after a short amount of time.
%
\begin{figure}[htb]
	\centering
	\includegraphics[scale=0.9]{img/NAPhaseStability}
	\caption{Phase deviation vs. averaging time $\tau$} \label{fig:NAPhaseStability}
\end{figure}
%

%
\begin{figure}[htb]
	\centering
	\includegraphics[scale=0.9]{img/NAAllanDev}
	\caption{Allan deviation} \label{fig:NAAllanDev}
\end{figure}
%
Figure \ref{fig:NAAllanDev} shows the Allan deviation. After averaging the fractional frequency over a time interval on the order of the duration of the bit pulse, $\tau = 10^{-10}$ s, the Allan deviation starts to fall off at the rate of $\tau^{-1}$. This falloff signifies that rapidly oscillating errors are being averaged out. We expect the falloff to continue indefinitely because XPM contributes no long-term frequency drift. We have computed the Allan deviation up to $10$ ns, at which point the trend proportional to $\tau^{-1}$ is apparent. Extrapolating the $\tau^{-1}$ dependence to longer averaging times, we find that the Allan deviation is $3\times 10^{-15}$ at $\tau = 1$ s, and $3\times 10^{-18}$ at $\tau=10^3$ s.

\section{Effect of Attenuation}

We now include the effect of attenuation. We expect the results to be lower than the phase and frequency stability without attenuation because the effective length before the nonlinearity, and hence XPM, becomes negligible is $20$ km after each amplifier.

%
\begin{figure}[htb]
	\centering
	\includegraphics[scale=0.9]{img/ACalcVar}
	\caption{Data channel optical power variance vs. fiber length} \label{fig:ACalcVar}
\end{figure}
%
First, we compare the variance of the attenuated data signal with the variance without attenuation. Figure~\ref{fig:ACalcVar} shows the data signal's optical power variance. Spikes occur every $80$ km, corresponding to the locations of the amplifiers.

%
\begin{figure}[htb]
	\begin{tabular}{c c}
		\includegraphics[width=0.5\linewidth]{img/APhiMean} & \includegraphics[width=0.5\linewidth]{img/APhiVar} \\
		(a) & (b)
	\end{tabular}
	\caption{(a)\label{fig:APhiMean} Mean of $\phi$ vs. fiber length (b)\label{fig:APhiVar} Variance of $\phi$ vs. fiber length}
\end{figure}
%
The mean and variance of $\phi$ must also grow over the length of the fiber, but they no longer grow almost linearly because the data signal power varies. Instead, the mean and the variance grow in steps. Figures \ref{fig:APhiMean}a and \ref{fig:APhiVar}b show the mean and variance of $\phi$ respectively, in which flat regions where the data signal power is low are visible. The mean and variance are less than those that we obtained when attenuation was neglected because average power of the data channels is lower.

%
\begin{figure}[htb]
	\centering
	\includegraphics[scale=0.9]{img/APhaseStability}
	\caption{Phase deviation with attenuation} \label{fig:APhaseStability}
\end{figure}
% 
The phase deviation $\left[D^{(1)}_\phi\right]^{1/2}$ reaches an asymptotic value as was the case when attenuation is neglected. Since the asymptotic value depends on the pulse spreading due to dispersion, the asymptotes occur at the same times. However, the phase deviation will be lower than was the case without attenuation, because the effect of XPM on the frequency signal from the data channels depends on the optical power of the data channels. Figure~\ref{fig:APhaseStability} shows $\left[D^{(1)}_\phi\right]^{1/2}$ for different fiber lengths.

%
\begin{figure}[htb]
	\centering
	\includegraphics[scale=0.9]{img/AAllanDev}
	\caption{Allan deviation with attenuation} \label{fig:AAllanDev}
\end{figure}
% 
Finally, the Allan deviation will be comparable to the results in the previous section. Figure~\ref{fig:AAllanDev} shows the Allan deviation for several fiber lengths, and we see a similar trend to what we saw without attenuation. At very low averaging times, $\tau < 10^{-11}$ s, there is higher uncertainty than in Figure~\ref{fig:NAAllanDev} due to high frequency ASE noise in the data signal. The Allan deviation will also decrease at a rate of $\tau^{-1}$ after an averaging time interval approximately equal to the bit duration. We perform the same extrapolation as in the previous section, and we find an Allan deviation of $10^{-15}$ at $\tau=1$ s, and $10^{-18}$ at $\tau=10^3$ s.

\section{Varying the Frequency Separation}

The frequency separation between the data channels and the frequency signal will lead to a group velocity difference between the data channels and the signal channels, due to chromatic dispersion. The relative group velocity difference governs the rate at which the data signal travels through a fixed time point in the frequency signal. The group velocity difference is related to the separation between the center frequencies of the data channels and the frequency signal. The value $\delta = (v_{f}-v_{d})/(v_fv_d) = 1$ ps/km is chosen because it corresponds to placing the frequency signal at the midpoint between two neighboring data channels separated by $12.5$ GHz. As the separation between the center frequencies decreases, the group velocity difference decreases and thus $\delta$ decreases. 

\begin{figure}[htb]
	\centering
	\includegraphics[scale=0.9]{img/GVPhaseStability}
	\caption{Phase deviation vs. group velocity difference} \label{fig:GVPhaseStability}
\end{figure}
%
Figure~\ref{fig:GVPhaseStability} shows the phase deviation for different frequency spacings between the data and frequency signal. The $6.25$ GHz separation refers to the system where the frequency signal is placed in the boundaries of two neighboring data signals. The $12.5$ GHz separation corresponds to placing the frequency signal in the neighboring channel ordinarily used for data. Further separations are used to demonstrate the effect of XPM diminishing as the data channel is placed further away from the frequency signal.

\section{Chapter Remarks}

Since the bits in any data signal are uncorrelated, the phase deviation will asymptote at an averaging time that corresponds approximately to the time that it takes a single bit to slide through a constant phase point in the frequency signal. When the relative group velocity is greater, the stability reaches its final value at a smaller distance because the data signal passes through a fixed time point in the frequency signal at a much faster rate. 

The Allan deviation represents the expected frequency error. Experiments performing frequency transfer with a frequency signal that occupies an entire data channel on the ITU grid have Allan deviations that are comparable to our simulated values \cite{Serrano2013,cantin2017progress}. In this case, the frequency signal and data channels are separated by many GHz. The source of error in the experiments is due to temperature fluctuations. We have found that placing a frequency signal in the interstices of two data signals gives a frequency error on the order of environmental effects and should therefore be feasible. Hence, it is not necessary to use an entire data channel to transfer a frequency signal.