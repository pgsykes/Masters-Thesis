\chapter{Results}
\label{chap:results}

\section{Introduction}

In the previous section, we presented limitations on the parameters of a commercial optical communication system that would limit the distortion on a frequency signal located on the interstices of two data channels in a DWDM scheme. The result was a phase distortion due to cross-talk (XPM) in the form
%
\begin{equation}
\phi(z,T) = 4\gamma\int_0^z |u_d(0, T-d\zeta)|^2 d\zeta.
\end{equation}
%
This shows that the phase distortion is dependent on the length of fiber, the relative velocity difference between the frequency signal and the data channels, and the power of the data channel while it changes over the course of its propagation. We can now vary these parameters and investigate their affect on the frequency stability of the frequency signal.

We begin by investigating the distribution of the power of the data channel, $|u_d|^2$, as a function of length, $z$. The pulses of the data channel change over the length due to dispersion, self-phase modulation, and attenuation. The length of the fiber is chosen so that we can simulate the dispersive effects and neglect the nonlinear effects. We first simulate without attenuation to emphasize the pulse shape due to spreading. Then attenuation is added. Next we change the group velocity difference. Then we vary the data channel powers.

\section{Simulation Parameters}

The simulation data signal is a $2^{10}-1$ pseudorandom binary string that is on-off key modulated with optical power of $1$ mW with periodic boundary conditions. The fiber has an attenuation of $0.2$ dB/km, group velocity dispersion $\beta_2 = -22$ ps$^2$/km, and Kerr nonlinearity $1.3$ W$^{-1}$km$^{-1}$. The data channel has a center wavelength of $1.5301389$ $\mu$m with group velocity difference $d = 1.003$ ps/km relative to the frequency signal. Some of these parameters will vary in the following sections as we study the changes in the XPM induced phase distortion.

\section{Without Attenuation}

During propagation, each pulse in the data signal is going to spread outside of its bit window into its neighbors. After some long distance, the energy in each pulse will spread evenly amongst every bit. The variance of the data signal 

\section{With Attenuation}

\section{Varying the Group Velocity Difference}

\section{Varying Data Channel Powers}



\section{Chapter Remarks}