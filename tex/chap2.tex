
	The individual data channels are modulated to transmit information.  Modulation formats are on-off keying (OOK), binary phase shift keying (BPSK), quadrature phase shift keying (QPSK), and differential phase shift keying (DPSK) \cite{agrawal2012fiber}.  An OOK signal is the simplest example, a binary $0$ is represented as the absence of power and a binary $1$ is some non-zero power above a threshold to distinguish a signal from noise.  There cannot be a sharp transition from a $0$ to a $1$ and vice versa, because a communication channel can only occupy a limited bandwidth.  In practice, each bit occupies a window of time where its value is held for a short time.  The signal can start building up to a $1$ in its preceding bit window and decay back to a $0$ in its following bit window.  The bits overlap into their neighboring windows and the amount of overlap is characterized by a roll-off parameter.  This overlap is known as intersymbol interference
