\chapter{Conclusion}

A frequency signal in an optical fiber suffers optical impairments due to the medium. By placing reasonable limits on the bandwidth, optical power, and frequency placements of the frequency signal, we make all of these impairments negligible except for cross-phase modulation.

Current systems that transfer frequencies over optical fibers either use dark fibers or use an entire data channel with a bandwidth of $10$ Gbps or $100$ Gbps. Since a frequency signal does not require large bandwidths, it is more efficient to place a frequency signal in between two data channels. However, placing the frequency signal close to the data channels will 
lead to phase and frequency errors. Limiting the errors will increase the commercial viability of frequency transfer using commercial optical fiber communication systems.

We computed the amount of frequency error due to XPM using the Allan deviation, and we found that it was comparable to environmental effects. Hence, it is feasible to place a frequency signal between two data channels on the ITU grid without a significant increase in errors due to cross-phase modulation.

\section{Future Work}

In our work to date, we did not take into account self-phase modulation (SPM) of the data channels. The parameters in modern communication systems are chosen to minimize the impact of SPM, and its largest effect is on the phase of a data channel, which has no impact on the frequency signal. Hence, its neglect is reasonable. Moreover, it is difficult computationally to study since we can no longer use Fourier transforms of the input data signal to compute its effect, but must solve the nonlinear Schr{\"o}dinger equation using a propagation code. Nonetheless, a careful investigation of its effect should be carried out at a future time.