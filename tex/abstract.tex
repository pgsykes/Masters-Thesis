%Abstract Page 

\hbox{\ }

\renewcommand{\baselinestretch}{1}
\small \normalsize

\begin{center}
\large{{ABSTRACT}} 

\vspace{2em} 

\end{center}
\hspace{-.15in}
\begin{tabular}{ll}
Title of dissertation:    & {\large  NONLINEAR INTERACTION BETWEEN }\\
&				      {\large  A FREQUENCY SIGNAL AND NEIGHBORING} \\
&				      {\large  DATA CHANNELS IN A COMMERCIAL} \\
&				      {\large  OPTICAL FIBER COMMUNICATION SYSTEM} \\
\ \\
&                          {\large  Patrick Sykes, Master of Science, 2018} \\
\ \\
Dissertation directed by: & {\large  Dr. Curtis Menyuk, Professor} \\
&  				{\large	 Computer Science and Electrical Engineering} \\
\end{tabular}

\vspace{2em}

\renewcommand{\baselinestretch}{2}
\large \normalsize

We theoretically investigate the feasibility of transmitting a frequency signal in an interstice of the data channels in a commercial wavelength division multiplexed optical fiber communications system.  
We will give an overview of some different measures used for frequency stability. 
We also list the typical optical impairments that affect light propagating in an optical fiber and how the impairments can induce phase noise in a frequency signal.
The phase noise on the frequency signal due to the optical impairments can be limited by restricting the optical power, bandwidth, and center frequency of the signal.
The primary source of phase noise is cross-phase modulation (XPM) between the frequency signal and its neighboring data channels.
We calculate the first order structure functions and Allan deviation of the phase noise resulting from XPM as the averaging time varies using typical commercial system parameters. 
We find that the instability added by this effect is comparable to experimentally observed instabilities in research networks, suggesting that frequency transfer over commercial networks without occupying an entire data channel should be feasible.
