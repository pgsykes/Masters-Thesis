\section{Introduction}

Time and frequency transfer is a less recognized necessity of modern society. Accurate timekeeping

Currently there are systems used for wireless time and frequency transfer in the form of satellites of various orbits. The advantages to using a fiber optic system are

There are projects for implementing optical fiber time and frequency transfer systems among various European research sites.  A notable project is the R\'eseau Fibr\'e M\'etrologique \`a Vocation Europ\'eenne (REFIMEVE+) project in France. And there is a larger project called CLONETS currently in investigation stage that aims to connect many European research institutes and high-technology companies with a time and frequency fiber system.  Widespread adoption of these systems motivates the re-definition of the second to be based on an optical comb lasers rather than current microwave cesium standard.














There are applications that require sharing a timing signal between devices.  One example is Global Positioning System (GPS) where a time signal is required to calculate geolocation.  Currently, there are techniques and systems for time and frequency transfer in wireless communication systems.  A notable example is the two-way satellite time and frequency transfer system which enables two laboratories to use a satellite as a common link to synchronize clocks.  The two laboratories will send and receive a time signal.  The received time signal is compared to their local clock and the difference between the two is the round trip delay.  Then, it is assumed that the delay is equal between both locations, so half of the round trip delay becomes the delay correction.  In another approach, each of two systems receives a time signal from the other system and also receives a common reference time signal from an independent source.  These wireless systems are typically accurate to within $1-10$ ns \cite{Allan1980}.  One major issue with a satellite system is the inaccessibility of the satellites, which makes hardware maintenance and upgrades difficult and the satellites themselves vulnerable to attack.  

Fiber optic communication systems are easy to access and can also transmit a timing signal. However, many impairments are introduced to the signal from the fiber \cite{agrawal2012fiber}.  

\begin{figure}[h!]
\label{fig:commfreq}
\centering
\begin{tikzpicture}
\draw [<->] (0,4) -- (0,0) -- (8.5,0);
\draw [thin] (0.25, 0) -- (2, 3.5) -- (3.75, 0);
\draw [thin] (4.25, 0) -- (4.25, 1.5);
\draw [thin] (4.75, 0) -- (6.5, 3.5) -- (8.25, 0);
\draw [dashed] (2, 0) -- (2,3.5);
\draw [dashed] (6.5, 0) -- (6.5,3.5);
\draw (8.7, 0) node {$f$};
\draw (2, -0.35) node {$f_1$};
\draw (4.25, -0.35) node {$f_2$};
\draw (6.5, -0.35) node {$f_3$};
\end{tikzpicture}
\caption{\label{fig:datatiming}Frequency domain representation of a communication system with two channels and a timing signal between them}
\end{figure}

In a typical optical communication system, there are many data channels centered at different optical wavelengths which is a technique called wavelength division multiplexing (WDM).  Each channel has some finite bandwidth so that they do not overlap in the frequency domain.  In this thesis, we will be considering the possibility of transmitting a timing signal in the interstices of the WDM channels.  We will examine the limits that fiber impairments impose on this timing signal in a long-haul system.  Signal impairments include amplified spontaneous emission (ASE) noise from amplifiers, dispersion, and the Kerr nonlinearity, scattering nonlinearities, Rayleigh, Brillouin, and Raman scattering, can also impair the signal \cite{agrawal2012fiber}\cite{agrawal1995nonlinear}.  Preliminary work indicates that this timing signal can be transmitted with both a narrow bandwidth ($< 100$ MHz) and low power ($<10$ $\mu W$) when compared to a data channel.  The bandwidth of a data channel in a long-haul system is typically $10-100$ GHz, while a typical power in a terrestrial long-haul system is $0$ dBm or $1$ mW at the transmitter and less than $-10$ dBm or $0.1$ mW prior to an amplifier as the signal attenuates.  In this case, the most important impairment that the timing channel suffers is due to cross-phase modulation between the timing channel and the neighboring data channels.  In this thesis, we will quantify the impact of cross-phase modulation on the timing channel and determine the limits that it imposes.


%We can introduce timing signals at some frequency with small power located between the limits of the neighboring data bands.  Signal impairments include amplified spontaneous emission (ASE) noise from amplifiers, dispersion, and the Kerr nonlinearity, scattering nonlinearities, Rayleigh, Brillouin, and Raman scattering, can also impair the signal \cite{agrawal2012fiber}\cite{agrawal1995nonlinear}.  Dispersion is the most important signal impairment.  Since the timing signal will be narrowband with small power, it is necessary to characterize what the fiber impairments will do to the signal and how the data signal will dominate the timing signal in the time domain.  Figure \ref{fig:datatiming} exhibits a rough idea of the multiplexed system's amplitude in the frequency domain.  The pyramids represent the data channels with center frequencies at f$_1$ and f$_3$.  The spike at $f_2$ represents the timing signal.

The individual data channels are modulated to transmit information.  Modulation formats are on-off keying (OOK), binary phase shift keying (BPSK), quadrature phase shift keying (QPSK), and differential phase shift keying (DPSK) \cite{agrawal2012fiber}.  An OOK signal is the simplest example, a binary $0$ is represented as the absence of power and a binary $1$ is some non-zero power above a threshold to distinguish a signal from noise.  There cannot be a sharp transition from a $0$ to a $1$ and vice versa, because a communication channel can only occupy a limited bandwidth.  In practice, each bit occupies a window of time where its value is held for a short time.  The signal can start building up to a $1$ in its preceding bit window and decay back to a $0$ in its following bit window.  The bits overlap into their neighboring windows and the amount of overlap is characterized by a roll-off parameter.  This overlap is known as intersymbol interference \cite{proakis2001digital}.

The timing signal will have periodic zero crossings.  However, nonlinearities from the fiber may alter the timing of the zero crossings.  This phase change causes the timing signal to appear as if it were different frequencies.  We must find the distribution of the amplitude of the data channel in order to calculate its variance and its impact on the phase change of the timing signal.  The distribution of the amplitude of the channel is mainly influenced by dispersion.  We calculate the effect of dispersion on an OOK signal after the signal has propagated a distance long enough for dispersion to completely spread it out.  The details of this computation are covered in the following sections.  Then an analytic result is derived and compared to the numerical results.










Chapter 2 introduces methods for measuring frequency stability of oscillators. We present the reasoning behind formulating different time stability measures, and the failures of usual statistical measures (mean, variance, etc.). Then we reveal the relations between each of the methods.

Chapter 3 covers a background of the impairments a signal experiences in an optical fiber.  The impairments will affect both a data channel and the timing channel. Here the limits of the impairments will be investigated which inform power and frequency requirements on the timing channel.

Chapter 4 contains the results of computational simulations. We derive the phase noise of the timing channel due to interactions with the data channel. Then we can perform time stability analysis on the phase noise to determine the amount of frequency fluctuation on the timing signal.

Chapter 5 concludes the thesis.


