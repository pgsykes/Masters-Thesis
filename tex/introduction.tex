\chapter{Introduction}

Improvements in optical frequency references allow them to be a better reference than current atomic clock standards. This opens up the potential to improve precision and accuracy in timekeeping systems as well as a redefinition of the second. That increase in precision and accuracy are limited by the synchronisation between those clocks using optical standards. There is some frequency uncertainty in a reference which becomes exacerbated as the reference is transmitted through a medium and influenced by the environment. Understanding the mechanisms for frequency distortion and controlling them where possible give stricter bounds on the frequency error.

Accurate timekeeping is required in the Global Positioning System (GPS) satellites and receivers, transaction logging, and research experiments. Currently, there are techniques and systems for time and frequency transfer in wireless communication systems.  A common method is the two-way satellite time and frequency transfer system which enables two laboratories to use a satellite as a common link to synchronize their clocks. These wireless systems are typically accurate to within $1-10$ ns \cite{Allan1980}.  Beyond improving the time accuracy, there is a significant issue of inaccessibility with satellite systems, which makes hardware maintenance and upgrades difficult and the satellites themselves vulnerable to attack.
Fiber optics can become a substitute for transfer, especially if one can take advantage of the existing fiber telecommunications infrastructure.


Research networks are increasingly transmitting time and frequency signals along with data over fiber optic communication systems. These networks include R\'eseau Fibr\'e M\'etrologique \`a Vocation Europ\'eenne (REFIMEVE+) in France \cite{cantin2017progress}, PIONIER in Poland \cite{Turza2017}, 
and the White Rabbit networks used for CERN accelerator sites and GSI's Facility for Antiproton and Ion Research \cite{Serrano2013}.  
A larger European optical time and frequency distribution network called CLONETS is planned \cite{CLONETS}.
The REFIMEVE+ project has demonstrated frequency transfer over optical fiber with a stability of $10^{-16}$ at $1$ s and $10^{-19}$ at $10^4$ s over a distance of $1480$ km \cite{cantin2017progress}. These systems place the frequency signal on a center frequency assigned for data transmissions. 
%This means that a large bandwidth of $12.5$ GHz to $100$ GHz \cite{ITU-T2012} is occupied by a narrowband ($\approx$MHz) frequency signal. We will investigate the feasibility of placing the frequency signal in the small gap between channels on the boundary of their bandwidths.

Many impairments are introduced to the signal from the fiber \cite{agrawal2012fiber}. In a typical optical communication system, there are many data channels centered at different optical wavelengths which is a technique called wavelength division multiplexing (WDM).  Each channel has some finite bandwidth so that they do not overlap in the frequency domain.  In this thesis, we will be considering the possibility of transmitting a frequency signal in the interstices of the WDM channels.  We will examine the limits that fiber impairments impose on this frequency signal in a long-haul system.  Signal impairments include amplified spontaneous emission (ASE) noise from amplifiers, dispersion, and the Kerr nonlinearity, scattering nonlinearities, Rayleigh, Brillouin, and Raman scattering, can also impair the signal \cite{agrawal2012fiber}\cite{Agrawal2013}.  Preliminary work indicates that this frequency signal can be transmitted with both a narrow bandwidth ($< 100$ MHz) and low power ($<10$ $\mu W$) when compared to a data channel.  The bandwidth of a data channel in a long-haul system is typically $10-100$ GHz \cite{ITU-T2012}, while a typical power in a terrestrial long-haul system is $0$ dBm or $1$ mW at the transmitter and less than $-10$ dBm or $0.1$ mW prior to an amplifier as the signal attenuates.  In this case, the most important impairment that the frequency signal suffers is due to cross-phase modulation between the frequency signal and the neighboring data channels.  In this thesis, we will quantify the impact of cross-phase modulation on the frequency signal and determine the limits that it imposes.


The individual data channels are modulated to transmit information.  Modulation formats are on-off keying (OOK), binary phase shift keying (BPSK), quadrature phase shift keying (QPSK), and differential phase shift keying (DPSK) \cite{agrawal2012fiber}.  An OOK signal is the simplest example, a binary $0$ is represented as the absence of power and a binary $1$ is some non-zero power above a threshold to distinguish a signal from noise.  There cannot be a sharp transition from a $0$ to a $1$ and vice versa, because a communication channel can only occupy a limited bandwidth.  In practice, each bit occupies a window of time where its value is held for a short time.  The signal can start building up to a $1$ in its preceding bit window and decay back to a $0$ in its following bit window.  The bits overlap into their neighboring windows and the amount of overlap is characterized by a roll-off parameter.  This overlap is known as intersymbol interference \cite{proakis2001digital}.

The frequency signal will have periodic zero crossings.  However, nonlinearities from the fiber may alter the timing of the zero crossings.  This phase change causes the frequency signal to appear as if it were different frequencies.  We must find the distribution of the amplitude of the data channel in order to calculate its variance and its impact on the phase change of the frequency signal.  The distribution of the amplitude of the channel is mainly influenced by dispersion.  We calculate the effect of dispersion on an OOK signal after the signal has propagated a distance long enough for dispersion to completely spread it out.


Chapter 2 introduces methods for measuring frequency stability of oscillators. We present the reasoning behind formulating different time stability measures, and the failures of usual statistical measures (mean, variance, etc.). Then we reveal the relations between each of the methods. We give a recommendation of the second structure function and Allan deviation for the measures of phase and frequency stability, respectively.

Chapter 3 is an overview of the common impairments a signal experiences in an optical fiber.  The impairments will affect both a data channel and the frequency signal. Here the limits of the impairments will be investigated which will inform power and frequency requirements on the frequency signal. After eliminating the negligible impairments, we highlight cross-talk as the principal cause of phase distortion in the frequency channel.

Chapter 4 contains the results of the phase noise computations.  We perform statistical and time stability analysis on the phase noise to determine the amount of frequency fluctuation on the frequency signal.

Chapter 5 concludes the thesis.


