\chapter{Introduction}

Improvements in optical frequency references allow them to be more precise than current atomic clock standards at microwave frequencies \cite{ProgressOpticalClock, RedefinitionOfSecond, Diddams2004}. It is expected that this greater precision will ultimately lead to a redefinition of the second, and greater precision and accuracy in timekeeping \cite{RedefinitionOfSecond}. It is desirable for many applications to transmit time and frequency from highly accurate and precise references, like those at the National Institute of Standards and Technology (NIST) or the US Naval Observatory (USNO), to distant locations. However, the transmission medium distorts the time and frequency data---degrading their accuracy and precision.


Numerous systems require accurate timekeeping, including Global Positioning System (GPS) satellites and receivers, transaction logging, and some basic science experiments \cite{ScienceOfTimekeeping}. Techniques and systems exist for time and frequency transfer in wireless communication systems.  A commonly-used method is two-way satellite time and frequency transfer, which makes it possible for two laboratories to use a satellite as a common link to synchronize their clocks. These wireless systems are typically accurate to within $1$--$10$ ns \cite{Allan1980}, which is sufficient for many applications, but far less accurate than the best primary references \cite{NISTFreqStandards, Audoin1976}. Additionally, satellites are physically inaccessible, which makes hardware maintenance and upgrades difficult and alse makes the satellites vulnerable to attack. Fiber optics are a potential substitute for land-based transfer, especially if one can take advantage of the existing fiber telecommunications infrastructure.


Research networks are increasingly transmitting time and frequency signals along with data over fiber optic communication systems. These networks include the R\'eseau Fibr\'e M\'etrologique \`a Vocation Europ\'eenne (REFIMEVE+) in France \cite{cantin2017progress}, PIONIER in Poland \cite{Turza2017}, 
and the White Rabbit networks used at the CERN accelerator sites and GSI's Facility for Antiproton and Ion Research \cite{Serrano2013}.  
A larger European optical time and frequency distribution network called CLONETS is planned \cite{CLONETS}.
The REFIMEVE+ project has demonstrated frequency transfer over optical fibers with a stability of $10^{-16}$ at $1$ s and $10^{-19}$ at $10^4$ s over a distance of $1480$ km \cite{cantin2017progress}. These systems place the frequency signal in a wavelength channel that is used for data transmission.
%This means that a large bandwidth of $12.5$ GHz to $100$ GHz \cite{ITU-T2012} is occupied by a narrowband ($\approx$MHz) frequency signal. We will investigate the feasibility of placing the frequency signal in the small gap between channels on the boundary of their bandwidths.

In a typical optical communication system, there are many data channels centered at different optical wavelengths, which is a technique called wavelength division multiplexing (WDM).  Each channel has some finite bandwidth so that they do not overlap in the frequency domain.  In this thesis, we will be considering the possibility of transmitting a frequency signal in the interstices of the WDM channels. We will examine the limits that fiber impairments impose on this frequency signal in a long-haul system.  A number of different physical effects impair signal transmission in optical fibers \cite{agrawal2012fiber}. Signal impairments include amplified spontaneous emission (ASE) noise from amplifiers, dispersion, and the Kerr nonlinearity \cite{agrawal2012fiber}.  Scattering nonlinearities due to the Rayleigh, Brillouin, and Raman effects, can also impair the signal \cite{agrawal2012fiber,Agrawal2013}.  Preliminary work indicates that this frequency signal can be transmitted with both a narrow bandwidth ($\lesssim 100$ MHz) and low power ($\lesssim 10$ $\mu$W) compared to a data channel \cite{menyukIFCS2015}.  The bandwidth of a data channel in a long-haul system is typically $10-100$ GHz \cite{ITU-T2012}, while a typical power in a terrestrial long-haul system for a WDM channel is $0$ dBm ($1$ mW) at the transmitter and less than $-10$ dBm ($0.1$ mW) prior to an amplifier as the signal attenuates.  In this case, the most important impairment that the frequency signal suffers is due to cross-phase modulation between the frequency signal and the neighboring data channels.  In this thesis, we will quantify the impact of cross-phase modulation on the frequency signal and determine the limits that it imposes.


The individual data channels are modulated to transmit information.  Examples of modulation formats are on-off keying (OOK), binary phase shift keying (BPSK), quadrature phase shift keying (QPSK), and differential phase shift keying (DPSK) \cite{agrawal2012fiber}.  An OOK signal is the simplest modulation format. A binary $0$ is represented by the absence of power in a time slot and a binary $1$ is represented by some non-zero power that is sufficiently large so that noise does not lead to an unacceptable probability of confusing $0$'s and $1$'s.  There cannot be a sharp transition from a $0$ to a $1$ and vice versa because a communication channel can only occupy a limited bandwidth.  In practice, each bit occupies a time slot where its value is held for a short time.  The signal can start building up to a $1$ from a $0$ in the preceding time slot and then decay back to a $0$ in the following time slot. Thus, the physical representation of the bits overlaps with neighboring bits, and the amount of overlap is characterized by a roll off parameter.  This overlap can lead to intersymbol interference \cite{proakis2001digital}.

A frequency signal has periodic zero crossings.  However, fiber impairments can alter the timing of the zero crossings.  These phase shifts broaden the frequency that is transmitted so that it is no longer a pure tone. We will show that the most important optical impairment is due to cross-phase modulation between a frequency signal and neighboring data channels. We then find the distribution of the amplitude of the data channels in order to calculate their variance and their impact on the variance of the frequency signal.  The distribution of the amplitude of the channel is mainly influenced by dispersion.  We calculate the effect of dispersion on an OOK signal, and we then calculate the variance of the data channel intensities as a function of distance. Given this variance, we can then calculate the phase and frequency variance of the frequency channel.


Chapter 2 introduces methods for measuring the frequency stability of oscillators. We present the reasoning behind different measures of time stability. We discuss issues with some of the usual statistical measures, such as the mean and  variance, that were developed for treating stationary processes. We reveal the relations between each of the measures. We focus in particular on the second structure function and Allan deviation for the measure of phase and frequency stability, respectively.

Chapter 3 is an overview of the common impairments a signal experiences in an optical fiber.  The impairments will affect both a data channel and the frequency signal. Here, the limits that the impairments impose on frequency transfer will be investigated. These impairments determine the power and frequency requirements for the frequency signal. After eliminating the negligible impairments, we demonstrate that cross-phase modulation is the principal non-environmental source of frequency spread in the frequency channel.

Chapter 4 describes the phase noise computations.  We perform statistical and time stability analyses on the phase noise to determine the variance of the frequency fluctuations.

Chapter 5 contains our conclusions and a discussion of future directions.


